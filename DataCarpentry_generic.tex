\documentclass[11pt]{article}
\usepackage[margin=1in]{geometry}
\usepackage[hyphens]{url}
\usepackage{parskip}
\setlength{\parindent}{0pt}

\usepackage{enumitem}
\setlist{nolistsep}
%\setlength{\skip\footins}{1cm}

\begin{document}


\title{Data Carpentry: \\workshops to increase data literacy for researchers\vspace{-1ex}}
\author{Tracy~K.~Teal\thanks{Michigan State University, East Lansing, MI, USA} \and Karen~A.~Cranston\thanks{National Evolutionary Synthesis Center (NESCent), Durham, NC, USA} \and Hilmar~Lapp\thanks{National Evolutionary Synthesis Center (NESCent), Durham, NC, USA} \and Ethan~White\thanks{Utah State University, Logan, UT, USA}\vspace{-2ex}}
%\thanks{(1) Michigan State University, East Lansing, MI, USA}
%\thanks{{}$^{(2)}$ National Evolutionary Synthesis Center (NESCent), Durham, NC, USA}
%\thanks{{}$^{(3)}$ Utah State University, Logan, UT, USA}
\date{June 10, 2014\vspace{-2ex}}



\maketitle


\section{Overview }
In many domains of science the rapid generation of large
amounts of data is fundamentally changing how research is done. The deluge of data presents great
opportunities, but also many challenges in managing, analyzing and
sharing data. Good training resources for researchers looking to develop
skills that will enable them to be more effective and
productive researchers are scarce. To address this need we have
developed an introductory bootcamp-style workshop, Data Carpentry,
designed to teach basic concepts, skills, and tools for working more
effectively with data.

Using the highly successful Software Carpentry bootcamps as a model,
we developed Data Carpentry as a two-day workshop that teaches basic
concepts, skills, and tools for working with data so researchers can
get more done in less time and with less pain. We modified existing
novice training materials from Software Carpentry,
developed new lessons, and integrated them in to a workshop focused on 
data and designed to
facilitate learning by researchers with little to no prior knowledge of
programming, shell scripting, and command line tools.

The first workshop was held at the National Evolutionary Synthesis
Center (NESCent) on May 8-9, 2014. 
Enthusiasm and support has already been overwhelming. The 30 seats
available for registration were gone in less than 24 hours, and
three times more students attempted to register than could be accommodated. Social media
support, discussion and interest has been extremely positive with
dozens of requests for workshops to be run, instructor
training, and material development. Based on the enthusiasm and
interest from the community, we are eager to continue developing
materials, to expand them to include domains beyond biology, and to
create a model scalable enough so we can offer these workshops at
locations throughout the world.
\\ 
\\ 

\section{Data Carpentry workshops to meet data training needs} \label{sec:dc-workshops}

Many organizations and groups have been working to develop a Data
Carpentry course, including
ELIXIR-UK\footnote{\url{http://elixir-uk.org/}},
ANDS\footnote{Australian National Data Service;
  \url{http://www.ands.org.au/}}, and BIO CollabIT\footnote{An
  informal consortium of science-supporting IT groups at
  interdisciplinary centers funded by the NSF BIO directorate;
  \url{http://www.nescent.org/wg_collabsci/}}) due to the shared need
for better training researchers in advanced data management, analysis,
and computational literacy. At a BIO CollabIT meeting in September
2013, informatics staff from several of the represented
interdisicplinary science centers developed the cornerstones of a data
and computational literacy workshop based on the Software Carpentry
model. Many of the attendees had taken or taught Software Carpentry
workshops, and had seen firsthand the success of this model.

Based on our experiences supporting researchers engaged in
interdisciplinary collaborative science, and informed by surveys
conducted across the NSF BIO-funced centers, we defined the following
overall learning objective for the course:
\begin{quote}{\emph{\textbf{Learning objective:} Researchers should be
      able to retrieve, view, manipulate, analyze and store their and
      other's data in an open and reproducible way.}}
\end{quote}

To attain this objective, we identified the following teaching subjects.
\begin{itemize}
\item How to use spreadsheet programs (such as Excel) more effectively, and the limitations of such programs.
\item Getting data out of Excel and into more powerful tools --- using R or Python.
\item Using databases, including managing and querying data in SQL.
\item Workflows and automating repetitive tasks, in particular using the command line shell and shell scripts.
\end{itemize}

In addition to the above subjects, the following skills emerged as
particularly important to impart from our discussions about designing
the course:
\begin{itemize}
\item Preparing data for analysis.
\item Using data and computational resources, in particular publicly available ones such as Amazon Web Services or iPlant Atmosphere.
\item Conducting data and computation-heavy research more reproducibly and openly.
\end{itemize}

Although the topics for Data Carpentry overlap substantially with
those for Software Carpentry, the Data Carpentry workshop differs in
its focus, its level of expected knowledge and its domain specificity.
\begin{itemize}
\item \emph{Data Carpentry is focused on data}. The workshop introduces one data set at the beginning of the
workshop. This data set is used throughout the workshop to teach how to manage and analyze data in an effective and reproducible way.
\item \emph{Data Carpentry is designed for novices.} There are no prerequisites, and no 
prior knowledge about the tools is assumed.
\item \emph{Data Carpentry is domain specific by design.} Researchers learn better when the example used is
of a kind they are familiar with. Learners can more easily integrate new skills and information into an existing framework, and 
are more motivated by example data similar to data encountered in their own work.
\end{itemize}

\section{Overview and interest in the first Data Carpentry workshop}

The first Data Carpentry workshop took place at NESCent May 8-9, 2014\footnote{Course website at \url{http://nescent.github.io/2014-05-08-datacarpentry/}}. Instructors who developed and taught the material were from NESCent
(HL and KAC), BEACON (TKT) and Utah State University (EW). Assistants, most of whom are planning to run and teach Data Carpentry courses themselves in the near future, were  from SESYNC\footnote{\url{http://sesync.org}} (Dr. Mike Smorul), iDigBio\footnote{\url{http://idigbio.org}} (Deborah Paul
and Matt Collins) and iPlant\footnote{\url{http://iplantcollaborative.org}} (Darren Boss).

The course provided room for 30 learners, and it was full within 3 hours of opening registration. An additional 64 people registered for the wait list, and anecdotal evidence suggests that there were many people who were interested but did not register for the wait list once they saw that the course was full. This immediate interest demonstrates the degree of need for this type of course.

The workshop started with learners describing why they were taking the course. The reasons included frustration with current data management and analysis approaches; an interest in advancing their research; teaching the tools to others; and learning the skills required for future career goals. The following examples illustrate the range of motivations:

\begin{itemize}
\item \emph{I'm tired of feeling out of my depth on computation and want to increase my confidence.}
\item \emph{I usually manage data in Excel and it's terrible and I want to do it better.}
\item \emph{I'm trying to reboot my lab's workflow to manage data and analysis in a more sustainable way.}
\item \emph{I want to use public data.}
\item \emph{I work with faculty at undergrad institutions and want to teach data practices, but I need to learn it myself first.}
\item \emph{I'm interested in going in to industry, and companies are asking for data analysis experience.}
\item \emph{I'm re-entering data over and over again by hand, and I know there's a better way.}
\item \emph{I have overwhelming amounts of next-generation sequencing data.}
\end{itemize}

This first course taught the topics outlined in Section~\ref{sec:dc-workshops}. Overall the workshop was well received, with positive comments to instructors and on social media. Post-assessment ratings gave the course an average rating of 8.25 out of 10. The experience of teaching the material and feedback from learners also gave rise to ways of refining both the topics taught and the order in which they are taught, as discussed in a post-workshop write-up\footnote{\url{http://software-carpentry.org/blog/2014/05/our-first-data-carpentry-workshop.html}}.

Broader interest in the workshop has been expressed through email to instructors, over Twitter and in blog post responses from researchers in biology, digital humanities, library sciences, and social sciences.
Interests have  
ranged from hosting future workshops, to teaching workshops and helping to develop materials.

Librarians and people working at university libraries have been particularly interested in Data Carpentry. With recent data management requirements from the NSF, NIH and other funding agencies, university libraries have
taken on the challenge of helping researchers develop data management plans, track data provenance and 
distribute and share data. Many libraries provide multiple resources and are actively developing or running workshops
on these topics, so Data Carpentry workshops fit well with their interest and engagement model.

This workshop also aligns well with training activity proposals labeled ``Data Carpentry'' from ELIXIR-UK. ELIXIR's goal is to ``build a sustainable European infrastructure for biological information, supporting life science research and its translation to medicine, agriculture, bioindustries and society'', and there is potential for mutual engagement between ELIXIR-UK and our group on the further development of this course. 

\section{Next steps for Data Carpentry}

Building off the enthusiasm and momentum of Data Carpentry, we want to continue training researchers in good data analysis and management practices. To scale up the number and breadth of researchers we can reach, we also want to move forward with more workshops being held, and more organizations participating in hosting, teaching, and material development.

\subsection{Initial next steps}

As the initial design of the workshop resulted from data management training gaps identified as common among the 
NSF BIO-funded science centers, running more workshops at other centers is the logical next step for Data Carpentry.
This will help further polish the materials so that the course meets learners' needs at different centers, in different
fields of (biological) science. Many of the instructors and assistants
for the NESCent workshop have since continued to refine and develop the materials taught, and the workflow in which they were taught.

The next such workshop is already scheduled at BEACON for July
24-25. TKT (BEACON) will be serve as one of the instructors, and is also training existing Software Carpentry instructors on teaching the Data Carpentry materials. 

\subsection{What would success for Data Carpentry look like}

\hangindent=0.7cm Goal: Develop and teach Data Carpentry workshops to help train the next generation of researchers in good data analysis and management practices to enable individual research progress and open and reproducible research. 

The challenge is to meet this need for data management training in many different locations and across multiple domains of interest.
In the future, if a researcher wants to take a Data Carpentry course, our vision is that such a course will be available to them within a reasonably time, or at a reasonable travel distance.  
 
Achieving this aim implies several long term goals that need to be met:

\begin{itemize}
\item The ability to host workshops in many different locations, including running
them multiple times in the same location 
\item Materials developed for different domains of interest 
\item Materials in both R and Python 
\item A streamlined assessment that lets us evaluate learning in a workshop and its effects
as researchers progress through their careers
\item A community forum for continued engagement on Data Carpentry topics post-workshop.
\item A set of resources where learners could look for more information on particular topics.
\item Workshops on more advanced topics - data visualization, statistics in R or Python
\end{itemize}

\textbf{What will be needed to achieve these goals?}

Several components are already in place or under active development:
\begin{itemize}
\item Public version control repositories for the development and distribution of materials (on Github).
\item Tutorials on topics that overlap with Software Carpentry, as many do.
\item First set of Data Carpentry-tailored materials, adapted from Software Carpentry.
\item Kick-off logistical support from Software Carpentry and the Mozilla Science Labs.
\end{itemize}

The components and capabilities yet to be established include the following:
\begin{itemize}
\item Personnel for establishing workshop guidelines and structure
\item Personnel for materials development and coordinating efforts
\item Train-the-Trainers workshops to expand the group of instructors, to increase the number of institutions with local trainers, 
and to decrease the requirement for instructor travel
\item The development of assessment materials and mechanisms to measure learning in workshops and to assess longer term impacts on researcher's data practices
\end{itemize}

One idea is to operate Data Carpentry similar to a franchise, with a strong Train-the-Trainers component that would enable
instructors to run workshops under the Data Carpentry brand at their local institutions, with central logistical support for registration, coordination of material development, and other tasks that benefit substantially from economies of scale.
Aiming for instructors to be typically local to where a course is run also reduces instructor coordination and travel costs.
People who expressed interest in teaching Data Carpentry tend to self-report as qualified to do so, provided they can 
receive some training on didactics and pedagogy. 
This is in contrast to what we typically observe for Software Carpentry, suggesting that Data Carpentry is
particularly well-suited to the franchise-model of scaling up.

\section{Acknowledgements}

The NESCent workshop could not have been developed or taught without the support of several organizations, including for personnel time, materials, and travel funds. Specifically, DataONE (NSF \#083094) supported the work of HL and KC on this workshop; NSF (NSF Career award) supported EW; BEACON (NSF) supported TKT; and BEACON and iDigBio (NSF) provided travel support to TKT and iDigBio personnel to serve as instructor and assistants, respectively, at the workshop.

We are also grateful to Software Carpentry and the Mozilla Science Lab for guidance on materials development and administrative support.

\end{document}.
